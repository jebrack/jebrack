% Options for packages loaded elsewhere
\PassOptionsToPackage{unicode}{hyperref}
\PassOptionsToPackage{hyphens}{url}
\PassOptionsToPackage{dvipsnames,svgnames,x11names}{xcolor}
%
\documentclass[
  11pt,
]{article}

\usepackage{amsmath,amssymb}
\usepackage{iftex}
\ifPDFTeX
  \usepackage[T1]{fontenc}
  \usepackage[utf8]{inputenc}
  \usepackage{textcomp} % provide euro and other symbols
\else % if luatex or xetex
  \usepackage{unicode-math}
  \defaultfontfeatures{Scale=MatchLowercase}
  \defaultfontfeatures[\rmfamily]{Ligatures=TeX,Scale=1}
\fi
\usepackage[]{Inter}
\ifPDFTeX\else  
    % xetex/luatex font selection
\fi
% Use upquote if available, for straight quotes in verbatim environments
\IfFileExists{upquote.sty}{\usepackage{upquote}}{}
\IfFileExists{microtype.sty}{% use microtype if available
  \usepackage[]{microtype}
  \UseMicrotypeSet[protrusion]{basicmath} % disable protrusion for tt fonts
}{}
\makeatletter
\@ifundefined{KOMAClassName}{% if non-KOMA class
  \IfFileExists{parskip.sty}{%
    \usepackage{parskip}
  }{% else
    \setlength{\parindent}{0pt}
    \setlength{\parskip}{6pt plus 2pt minus 1pt}}
}{% if KOMA class
  \KOMAoptions{parskip=half}}
\makeatother
\usepackage{xcolor}
\usepackage[lmargin=1in,rmargin=1in,tmargin=1in,bmargin=1in]{geometry}
\setlength{\emergencystretch}{3em} % prevent overfull lines
\setcounter{secnumdepth}{-\maxdimen} % remove section numbering
% Make \paragraph and \subparagraph free-standing
\makeatletter
\ifx\paragraph\undefined\else
  \let\oldparagraph\paragraph
  \renewcommand{\paragraph}{
    \@ifstar
      \xxxParagraphStar
      \xxxParagraphNoStar
  }
  \newcommand{\xxxParagraphStar}[1]{\oldparagraph*{#1}\mbox{}}
  \newcommand{\xxxParagraphNoStar}[1]{\oldparagraph{#1}\mbox{}}
\fi
\ifx\subparagraph\undefined\else
  \let\oldsubparagraph\subparagraph
  \renewcommand{\subparagraph}{
    \@ifstar
      \xxxSubParagraphStar
      \xxxSubParagraphNoStar
  }
  \newcommand{\xxxSubParagraphStar}[1]{\oldsubparagraph*{#1}\mbox{}}
  \newcommand{\xxxSubParagraphNoStar}[1]{\oldsubparagraph{#1}\mbox{}}
\fi
\makeatother

\usepackage{color}
\usepackage{fancyvrb}
\newcommand{\VerbBar}{|}
\newcommand{\VERB}{\Verb[commandchars=\\\{\}]}
\DefineVerbatimEnvironment{Highlighting}{Verbatim}{commandchars=\\\{\}}
% Add ',fontsize=\small' for more characters per line
\usepackage{framed}
\definecolor{shadecolor}{RGB}{241,243,245}
\newenvironment{Shaded}{\begin{snugshade}}{\end{snugshade}}
\newcommand{\AlertTok}[1]{\textcolor[rgb]{0.68,0.00,0.00}{#1}}
\newcommand{\AnnotationTok}[1]{\textcolor[rgb]{0.37,0.37,0.37}{#1}}
\newcommand{\AttributeTok}[1]{\textcolor[rgb]{0.40,0.45,0.13}{#1}}
\newcommand{\BaseNTok}[1]{\textcolor[rgb]{0.68,0.00,0.00}{#1}}
\newcommand{\BuiltInTok}[1]{\textcolor[rgb]{0.00,0.23,0.31}{#1}}
\newcommand{\CharTok}[1]{\textcolor[rgb]{0.13,0.47,0.30}{#1}}
\newcommand{\CommentTok}[1]{\textcolor[rgb]{0.37,0.37,0.37}{#1}}
\newcommand{\CommentVarTok}[1]{\textcolor[rgb]{0.37,0.37,0.37}{\textit{#1}}}
\newcommand{\ConstantTok}[1]{\textcolor[rgb]{0.56,0.35,0.01}{#1}}
\newcommand{\ControlFlowTok}[1]{\textcolor[rgb]{0.00,0.23,0.31}{\textbf{#1}}}
\newcommand{\DataTypeTok}[1]{\textcolor[rgb]{0.68,0.00,0.00}{#1}}
\newcommand{\DecValTok}[1]{\textcolor[rgb]{0.68,0.00,0.00}{#1}}
\newcommand{\DocumentationTok}[1]{\textcolor[rgb]{0.37,0.37,0.37}{\textit{#1}}}
\newcommand{\ErrorTok}[1]{\textcolor[rgb]{0.68,0.00,0.00}{#1}}
\newcommand{\ExtensionTok}[1]{\textcolor[rgb]{0.00,0.23,0.31}{#1}}
\newcommand{\FloatTok}[1]{\textcolor[rgb]{0.68,0.00,0.00}{#1}}
\newcommand{\FunctionTok}[1]{\textcolor[rgb]{0.28,0.35,0.67}{#1}}
\newcommand{\ImportTok}[1]{\textcolor[rgb]{0.00,0.46,0.62}{#1}}
\newcommand{\InformationTok}[1]{\textcolor[rgb]{0.37,0.37,0.37}{#1}}
\newcommand{\KeywordTok}[1]{\textcolor[rgb]{0.00,0.23,0.31}{\textbf{#1}}}
\newcommand{\NormalTok}[1]{\textcolor[rgb]{0.00,0.23,0.31}{#1}}
\newcommand{\OperatorTok}[1]{\textcolor[rgb]{0.37,0.37,0.37}{#1}}
\newcommand{\OtherTok}[1]{\textcolor[rgb]{0.00,0.23,0.31}{#1}}
\newcommand{\PreprocessorTok}[1]{\textcolor[rgb]{0.68,0.00,0.00}{#1}}
\newcommand{\RegionMarkerTok}[1]{\textcolor[rgb]{0.00,0.23,0.31}{#1}}
\newcommand{\SpecialCharTok}[1]{\textcolor[rgb]{0.37,0.37,0.37}{#1}}
\newcommand{\SpecialStringTok}[1]{\textcolor[rgb]{0.13,0.47,0.30}{#1}}
\newcommand{\StringTok}[1]{\textcolor[rgb]{0.13,0.47,0.30}{#1}}
\newcommand{\VariableTok}[1]{\textcolor[rgb]{0.07,0.07,0.07}{#1}}
\newcommand{\VerbatimStringTok}[1]{\textcolor[rgb]{0.13,0.47,0.30}{#1}}
\newcommand{\WarningTok}[1]{\textcolor[rgb]{0.37,0.37,0.37}{\textit{#1}}}

\providecommand{\tightlist}{%
  \setlength{\itemsep}{0pt}\setlength{\parskip}{0pt}}\usepackage{longtable,booktabs,array}
\usepackage{calc} % for calculating minipage widths
% Correct order of tables after \paragraph or \subparagraph
\usepackage{etoolbox}
\makeatletter
\patchcmd\longtable{\par}{\if@noskipsec\mbox{}\fi\par}{}{}
\makeatother
% Allow footnotes in longtable head/foot
\IfFileExists{footnotehyper.sty}{\usepackage{footnotehyper}}{\usepackage{footnote}}
\makesavenoteenv{longtable}
\usepackage{graphicx}
\makeatletter
\newsavebox\pandoc@box
\newcommand*\pandocbounded[1]{% scales image to fit in text height/width
  \sbox\pandoc@box{#1}%
  \Gscale@div\@tempa{\textheight}{\dimexpr\ht\pandoc@box+\dp\pandoc@box\relax}%
  \Gscale@div\@tempb{\linewidth}{\wd\pandoc@box}%
  \ifdim\@tempb\p@<\@tempa\p@\let\@tempa\@tempb\fi% select the smaller of both
  \ifdim\@tempa\p@<\p@\scalebox{\@tempa}{\usebox\pandoc@box}%
  \else\usebox{\pandoc@box}%
  \fi%
}
% Set default figure placement to htbp
\def\fps@figure{htbp}
\makeatother

\makeatletter
\@ifpackageloaded{caption}{}{\usepackage{caption}}
\AtBeginDocument{%
\ifdefined\contentsname
  \renewcommand*\contentsname{Table of contents}
\else
  \newcommand\contentsname{Table of contents}
\fi
\ifdefined\listfigurename
  \renewcommand*\listfigurename{List of Figures}
\else
  \newcommand\listfigurename{List of Figures}
\fi
\ifdefined\listtablename
  \renewcommand*\listtablename{List of Tables}
\else
  \newcommand\listtablename{List of Tables}
\fi
\ifdefined\figurename
  \renewcommand*\figurename{Figure}
\else
  \newcommand\figurename{Figure}
\fi
\ifdefined\tablename
  \renewcommand*\tablename{Table}
\else
  \newcommand\tablename{Table}
\fi
}
\@ifpackageloaded{float}{}{\usepackage{float}}
\floatstyle{ruled}
\@ifundefined{c@chapter}{\newfloat{codelisting}{h}{lop}}{\newfloat{codelisting}{h}{lop}[chapter]}
\floatname{codelisting}{Listing}
\newcommand*\listoflistings{\listof{codelisting}{List of Listings}}
\makeatother
\makeatletter
\makeatother
\makeatletter
\@ifpackageloaded{caption}{}{\usepackage{caption}}
\@ifpackageloaded{subcaption}{}{\usepackage{subcaption}}
\makeatother

\usepackage{bookmark}

\IfFileExists{xurl.sty}{\usepackage{xurl}}{} % add URL line breaks if available
\urlstyle{same} % disable monospaced font for URLs
\hypersetup{
  pdftitle={Assignment 1: Data Wrangling Memo},
  pdfauthor={Jenna Bracken},
  colorlinks=true,
  linkcolor={blue},
  filecolor={Maroon},
  citecolor={Blue},
  urlcolor={Blue},
  pdfcreator={LaTeX via pandoc}}


\title{Assignment 1: Data Wrangling Memo}
\author{Jenna Bracken}
\date{}

\begin{document}
\maketitle


\section{Assignment 1: The Research ``Mise en
Place''}\label{assignment-1-the-research-mise-en-place}

\subsection{Conceptual Focus}\label{conceptual-focus}

This assignment covers \textbf{Chapter 12: Data Wrangling}. As your
slides note, this is the ``mise en place'' of research. Raw client data
is messy. Our ``Garbage in, garbage out'' motto means we must create a
clean, reliable, and reproducible workflow \emph{before} any analysis
can begin. Your R script is your ``precise, shareable, and repeatable
recipe'' that ensures your work is transparent and accurate.

\begin{center}\rule{0.5\linewidth}{0.5pt}\end{center}

\section{Data Prep Memo}\label{data-prep-memo}

\textbf{To:} Alex Chen (Director of Insights)

\textbf{From:} Jenna Bracken

\textbf{Subject:} Data Prep Memo for 2025 Digital Landscape Brief

To prepare the raw survey data for analysis, I performed a series of
transformations to ensure accuracy and consistency. First, I recoded
social media usage variables into binary indicators (uses\_facebook,
uses\_instagram, uses\_tiktok, uses\_x), where 1 represents active use
and 0 represents non-use or refusal. This step simplifies modeling and
interpretation, allowing us to easily calculate engagement metrics and
run logistic regressions later. Next, I created a composite variable
called platform\_count by summing these four binary indicators. This
provides a quick measure of how many platforms each respondent uses,
which is useful for segmenting audiences by overall social media
engagement. I also generated a simplified party affiliation variable
(party\_simple) by combining two original variables---party
identification and party leaning---into three categories: Republican,
Democrat, and Independent/Other. This categorical transformation ensures
clarity for reporting and aligns with the client's need for high-level
political segmentation. Finally, I built an index score
(fb\_purpose\_score) to capture the breadth of reasons respondents use
Facebook. By summing seven binary indicators for different purposes, we
created a scale ranging from 0 to 7, which reflects engagement depth
rather than mere presence. These transformations collectively turn messy
raw data into structured, analysis-ready variables that support clear
insights for the client presentation.

\newpage

\section{Appendix: R Code and
Commentary}\label{appendix-r-code-and-commentary}

This appendix contains all R code used to import, clean, and transform
the raw \texttt{ATP\_W144\_excerpt.csv} file into the analysis-ready
\texttt{w144\_wrangled.RData} file.

\subsection{1. Setup}\label{setup}

In RStudio, these options are commonly used to control how code and
output appear in the final report. The line \#\textbar{} label:
``chunk-setup'' assigns a label to the code chunk, making it easier to
reference later. The option \#\textbar{} echo: true ensures that the
code itself is displayed in the rendered document, which is helpful for
reporting. Meanwhile, \#\textbar{} message: false and \#\textbar{}
warning: false suppress messages and warnings from appearing in the
output, keeping the document clean and professional. Finally,
library(tidyverse) loads a collection of packages for data manipulation
and visualization.

\begin{Shaded}
\begin{Highlighting}[]
\FunctionTok{library}\NormalTok{(tidyverse)}
\end{Highlighting}
\end{Shaded}

\subsection{2. Importing}\label{importing}

You would use this code to import your dataset into R for analysis. The
chunk options \#\textbar{} label: ``chunk-import'' and \#\textbar{}
echo: true are Quarto settings that label the code chunk and ensure the
code is displayed in the rendered document, which is helpful for
organization and transparency. The function read\_csv() from the readr
package (part of the tidyverse) is used to load a CSV file---in this
case, ``ATP\_W144\_excerpt.csv''---into R as a data frame called
w144\_raw. This step is essential because you need the raw data in your
R environment before you can clean, transform, or analyze it.

\begin{Shaded}
\begin{Highlighting}[]
\NormalTok{w144\_raw }\OtherTok{\textless{}{-}} \FunctionTok{read.csv}\NormalTok{(}\StringTok{"ATP\_W144\_excerpt.csv"}\NormalTok{)}
\end{Highlighting}
\end{Shaded}

\subsection{3. Transforming: Binary Platform
Use}\label{transforming-binary-platform-use}

This code is used to transform raw survey data into a cleaner,
analysis-ready format by creating binary variables for social media
usage. The original variable \texttt{smuse\_a\_w144} uses codes where
\texttt{1} means ``Yes,'' \texttt{2} means ``No,'' and \texttt{99} means
``Refused.'' For statistical modeling and interpretation, it's common to
recode these into binary indicators, assigning \texttt{1} for ``Yes''
and \texttt{0} for ``No'' or ``Refused.'' The \texttt{mutate()} function
from the \texttt{dplyr} package is applied to the raw dataset
(\texttt{w144\_raw}) to add new columns such as \texttt{uses\_facebook},
which is defined using \texttt{if\_else()} to implement this logic.
Additional placeholders for Instagram, TikTok, and X are included so you
can apply the same transformation to those variables. This approach
ensures consistency, simplifies analysis, and makes the data suitable
for models like logistic regression or for generating clear summaries.

\begin{Shaded}
\begin{Highlighting}[]
\NormalTok{w144\_wrangled }\OtherTok{\textless{}{-}}\NormalTok{ w144\_raw }\SpecialCharTok{\%\textgreater{}\%}
  \FunctionTok{mutate}\NormalTok{(}
    \AttributeTok{uses\_facebook =} \FunctionTok{if\_else}\NormalTok{(smuse\_a\_w144 }\SpecialCharTok{==} \DecValTok{1}\NormalTok{, }\DecValTok{1}\NormalTok{, }\DecValTok{0}\NormalTok{),}
    \AttributeTok{uses\_instagram =} \FunctionTok{if\_else}\NormalTok{(smuse\_a\_w144 }\SpecialCharTok{==} \DecValTok{1}\NormalTok{, }\DecValTok{1}\NormalTok{, }\DecValTok{0}\NormalTok{), }
    \AttributeTok{uses\_tiktok =} \FunctionTok{if\_else}\NormalTok{(smuse\_a\_w144 }\SpecialCharTok{==} \DecValTok{1}\NormalTok{, }\DecValTok{1}\NormalTok{, }\DecValTok{0}\NormalTok{), }
    \AttributeTok{uses\_x =} \FunctionTok{if\_else}\NormalTok{(smuse\_a\_w144 }\SpecialCharTok{==} \DecValTok{1}\NormalTok{, }\DecValTok{1}\NormalTok{, }\DecValTok{0}\NormalTok{)}
\NormalTok{    )}
\end{Highlighting}
\end{Shaded}

\subsection{4. Transforming: Composite Scale
(platform\_count)}\label{transforming-composite-scale-platform_count}

This step creates a composite measure of social media engagement by
summing the four binary variables (uses\_facebook, uses\_instagram,
uses\_tiktok, uses\_x). Because each variable is coded as 1 for ``Yes''
and 0 for ``No,'' adding them together produces a simple count of how
many platforms each respondent uses. This is what our slides refer to as
a Mathematical Transformation: converting multiple indicators into a
single numeric scale that can be used for descriptive statistics or
modeling.

\begin{Shaded}
\begin{Highlighting}[]
\NormalTok{w144\_wrangled }\OtherTok{\textless{}{-}}\NormalTok{ w144\_wrangled }\SpecialCharTok{\%\textgreater{}\%}
  \FunctionTok{mutate}\NormalTok{(}
    \CommentTok{\# Hint: Add your four new uses\_... variables together.}
    \AttributeTok{platform\_count =}\NormalTok{ uses\_facebook }\SpecialCharTok{+}\NormalTok{ uses\_instagram }\SpecialCharTok{+}\NormalTok{ uses\_tiktok }\SpecialCharTok{+}\NormalTok{ uses\_x}
\NormalTok{  )}
\end{Highlighting}
\end{Shaded}

\subsection{5. Transforming: Categorical
(party\_simple)}\label{transforming-categorical-party_simple}

We combine f\_party and f\_partyln because respondents may identify with
a party or lean toward one, and both indicators are needed for accurate
classification. The case\_when() function applies logical conditions to
assign respondents into three categories: ``Republican,'' ``Democrat,''
or ``Independent/Other.'' This is a Categorical Transformation,
simplifying multiple numeric codes into meaningful labels for analysis
and reporting.

\begin{Shaded}
\begin{Highlighting}[]
\NormalTok{w144\_wrangled }\OtherTok{\textless{}{-}}\NormalTok{ w144\_wrangled }\SpecialCharTok{\%\textgreater{}\%}
  \FunctionTok{mutate}\NormalTok{(}
    \AttributeTok{party\_simple =} \FunctionTok{case\_when}\NormalTok{(f\_party\_final }\SpecialCharTok{==} \DecValTok{1} \SpecialCharTok{|}\NormalTok{ f\_partyln\_final }\SpecialCharTok{==} \DecValTok{1} \SpecialCharTok{\textasciitilde{}} \StringTok{"Republican"}\NormalTok{,}
\NormalTok{      f\_party\_final }\SpecialCharTok{==} \DecValTok{2} \SpecialCharTok{|}\NormalTok{ f\_partyln\_final }\SpecialCharTok{==} \DecValTok{2} \SpecialCharTok{\textasciitilde{}} \StringTok{"Democrat"}\NormalTok{,}
\NormalTok{      f\_party\_final }\SpecialCharTok{==} \DecValTok{3} \SpecialCharTok{|}\NormalTok{ f\_party\_final }\SpecialCharTok{==} \DecValTok{4}   \SpecialCharTok{\textasciitilde{}} \StringTok{"Independent/Other"}\NormalTok{,}
      \ConstantTok{TRUE} \SpecialCharTok{\textasciitilde{}} \ConstantTok{NA\_character\_}
\NormalTok{    )}
\NormalTok{  )}
\end{Highlighting}
\end{Shaded}

\subsection{6. Transforming: Index Score
(fb\_purpose\_score)}\label{transforming-index-score-fb_purpose_score}

This index measures the breadth of reasons respondents use Facebook.
Each fbwhy\_ variable represents a specific purpose, coded as 1 if
selected. By checking (fbwhy\_x\_w144 == 1) for each and adding them
together, we leverage R's behavior of treating TRUE as 1 and FALSE as 0.
The result is a composite score ranging from 0 to 7, indicating how many
distinct purposes a respondent reported. This is a Composite Index
Score, useful for understanding engagement depth.

\begin{Shaded}
\begin{Highlighting}[]
\NormalTok{w144\_wrangled }\OtherTok{\textless{}{-}}\NormalTok{ w144\_wrangled }\SpecialCharTok{\%\textgreater{}\%}
  \FunctionTok{mutate}\NormalTok{((fbwhy\_a\_w144 }\SpecialCharTok{==} \DecValTok{1}\NormalTok{) }\SpecialCharTok{+} 
\NormalTok{           (fbwhy\_b\_w144 }\SpecialCharTok{==} \DecValTok{1}\NormalTok{) }\SpecialCharTok{+}
\NormalTok{           (fbwhy\_c\_w144 }\SpecialCharTok{==} \DecValTok{1}\NormalTok{) }\SpecialCharTok{+}
\NormalTok{           (fbwhy\_d\_w144 }\SpecialCharTok{==} \DecValTok{1}\NormalTok{) }\SpecialCharTok{+}
\NormalTok{           (fbwhy\_e\_w144 }\SpecialCharTok{==} \DecValTok{1}\NormalTok{) }\SpecialCharTok{+}
\NormalTok{           (fbwhy\_f\_w144 }\SpecialCharTok{==} \DecValTok{1}\NormalTok{) }\SpecialCharTok{+}
\NormalTok{           (fbwhy\_g\_w144 }\SpecialCharTok{==} \DecValTok{1}\NormalTok{)}

\NormalTok{  )}
\end{Highlighting}
\end{Shaded}

\subsection{7. Exporting}\label{exporting}

We save the wrangled dataset as an .RData file because it preserves
R-specific data structures (such as factors and attributes) and is more
efficient for loading in future R sessions. Unlike .csv, which stores
plain text, .RData ensures that all variable types and metadata remain
intact, supporting reproducibility and consistency in analysis.

\begin{Shaded}
\begin{Highlighting}[]
\CommentTok{\# Save the final wrangled data}
\FunctionTok{save}\NormalTok{(w144\_wrangled, }\AttributeTok{file =} \StringTok{"w144\_wrangled.RData"}\NormalTok{)}
\end{Highlighting}
\end{Shaded}





\end{document}
